\documentclass[a4paper,12pt]{article}
\usepackage[utf8]{inputenc}
\usepackage{geometry}
\geometry{a4paper, margin=1in}
\usepackage{hyperref}
\usepackage{amsmath}

\title{Raccolta e Analisi dei Requisiti\\ \textbf{Climate Monitoring System}}
\author{}
\date{\today}

\begin{document}

\maketitle
\tableofcontents
\newpage

\section{Introduzione}
Il presente documento descrive la raccolta e l'analisi dei requisiti per lo sviluppo del sistema di monitoraggio climatico denominato \textbf{Climate Monitoring System (CM)}. Il sistema è progettato per raccogliere e rendere disponibili i dati climatici forniti da centri di monitoraggio situati in tutto il territorio italiano, permettendo agli operatori ambientali e ai cittadini comuni di consultare le informazioni relative alla loro zona di interesse.

\section{Descrizione Generale}
Il sistema CM si compone di due moduli principali: \textbf{serverCM}, che si interfaccia con un DBMS relazionale per fornire i servizi di back-end, e \textbf{clientCM}, che fornisce i servizi e le funzionalità per gli utenti finali. Il sistema supporta l'interazione in parallelo con più utenti connessi da postazioni differenti.

\subsection{Obiettivi del Sistema}
Gli obiettivi principali del sistema CM includono:
\begin{itemize}
    \item Fornire un repository centralizzato per la raccolta e la gestione dei dati climatici.
    \item Consentire agli operatori autorizzati di creare e gestire centri di monitoraggio e aree di interesse.
    \item Rendere disponibili i dati climatici ai cittadini comuni in forma aggregata.
\end{itemize}

\subsection{Utenti del Sistema}
Il sistema CM è progettato per due categorie di utenti:
\begin{itemize}
    \item \textbf{Operatori Autorizzati}: Possono registrarsi al sistema, creare centri di monitoraggio, inserire dati climatici e gestire aree di interesse.
    \item \textbf{Cittadini Comuni}: Possono consultare le informazioni climatiche delle aree di interesse senza necessità di registrazione.
\end{itemize}

\section{Requisiti Funzionali}
Di seguito sono elencati i principali requisiti funzionali del sistema CM.

\subsection{Gestione della Registrazione e Autenticazione}
\begin{itemize}
    \item \textbf{registrazione()}: Permette agli operatori di registrarsi al sistema inserendo nome, cognome, codice fiscale, indirizzo email, userid, password e centro di monitoraggio (se presente).
    \item \textbf{login()}: Permette agli operatori registrati di accedere al sistema tramite userid e password.
\end{itemize}

\subsection{Gestione delle Aree di Interesse}
\begin{itemize}
    \item \textbf{creaCentroMonitoraggio()}: Permette agli operatori registrati di creare centri di monitoraggio inserendo nome, indirizzo e l'elenco delle aree di interesse.
    \item \textbf{inserisciParametriClimatici()}: Consente agli operatori registrati di inserire i dati climatici (vento, umidità, pressione, temperatura, ecc.) per ciascuna area di interesse e per una specifica data di rilevazione.
\end{itemize}

\subsection{Consultazione delle Aree di Interesse}
\begin{itemize}
    \item \textbf{cercaAreaGeografica()}: Permette di cercare aree geografiche per denominazione o per coordinate geografiche.
    \item \textbf{visualizzaAreaGeografica()}: Consente di visualizzare le informazioni dettagliate e i dati climatici aggregati relativi a un'area di interesse.
\end{itemize}

\section{Requisiti Non Funzionali}
\begin{itemize}
    \item \textbf{Scalabilità}: Il sistema deve essere in grado di gestire un numero elevato di utenti connessi simultaneamente.
    \item \textbf{Sicurezza}: L'accesso ai dati sensibili deve essere protetto tramite autenticazione e autorizzazione.
    \item \textbf{Usabilità}: L'interfaccia utente deve essere intuitiva e facile da usare sia per operatori che per cittadini comuni.
\end{itemize}

\section{Design del Database}
Il sistema utilizza un database relazionale (PostgreSQL) per memorizzare i dati relativi alle aree di interesse, ai centri di monitoraggio, agli operatori registrati e ai parametri climatici. Di seguito sono descritte le tabelle principali.

\subsection{Tabella CoordinateMonitoraggio}
Contiene le informazioni sulle aree di interesse monitorate. I campi principali includono:
\begin{itemize}
    \item \textbf{Latitudine}
    \item \textbf{Longitudine}
    \item \textbf{Denominazione ufficiale}
    \item \textbf{Stato}
\end{itemize}

\subsection{Tabella CentriMonitoraggio}
Memorizza le informazioni sui centri di monitoraggio. I campi principali includono:
\begin{itemize}
    \item \textbf{Nome Centro Monitoraggio}
    \item \textbf{Indirizzo fisico}
    \item \textbf{Elenco delle aree di interesse}
\end{itemize}

\subsection{Tabella OperatoriRegistrati}
Contiene le informazioni sugli operatori autorizzati. I campi principali includono:
\begin{itemize}
    \item \textbf{Nome e cognome}
    \item \textbf{Codice fiscale}
    \item \textbf{Indirizzo email}
    \item \textbf{Userid}
    \item \textbf{Password}
    \item \textbf{Centro di monitoraggio di afferenza}
\end{itemize}

\subsection{Tabella ParametriClimatici}
Memorizza i dati climatici inseriti dagli operatori per ciascuna area di interesse e data di rilevazione. I campi principali includono:
\begin{itemize}
    \item \textbf{Centro di monitoraggio}
    \item \textbf{Area di interesse}
    \item \textbf{Data di rilevazione}
    \item \textbf{Parametri climatici (vento, umidità, pressione, ecc.)}
    \item \textbf{Commenti (opzionali)}
\end{itemize}

\section{Conclusioni}
Il documento ha presentato la raccolta e l'analisi dei requisiti per lo sviluppo del sistema CM. Una corretta implementazione del sistema richiederà un'attenta progettazione sia del backend che dell'interfaccia utente, al fine di garantire un'esperienza utente ottimale e la corretta gestione dei dati climatici.

\end{document}
