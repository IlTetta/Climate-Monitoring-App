\section{Strutture Dati e scelte Algoritmiche}
I dati vengono salvati in modo definitivo in un file \texttt{.csv}. Quando si apre l'applicazione, i dati contenuti nei file vengono letti e caricati in delle mappe strutturate
come relazioni in un database (chiave primaria + tupla), da cui poi estrarre i dati quando richiessto.\\
Abbiamo scelto di utilizzare le \texttt{HashMap} per diversi motivi. In primo luogo, l'uso di una \texttt{HashMap} in memoria offre un accesso rapido ed efficinete ai dati. 
Una volta che i dati sono stati letti dai file, vengono memorizzati in memoria sottoforma di \texttt{HashMap}, consentendo operazioni di accesso rapido con complessità $O(1)$.\\ 
Lavorare direttamente sui file avrebbe richiesto operazioni di lettura e scrittura più frequenti, coinvolgendo l'I/O del disco, che è generalmente più lento rispetto all'I/O in memoria.
Memorizzare i dati in una \texttt{HashMap} consente di evitare la necessità di aprire e chiudere frequentemente i file durante le operazioni.\\
Inoltre, le \texttt{HashMap} forniscono una struttura dati efficiente per la gestione di associazioi chiave-valore. Questo è particolarmente utile nel nostro caso, dova abbiamo bisogno
di eseguire operazioni di ricerca, inserimento e aggiornamento dei dati in modo veloce ed efficiente.\\
Alcune della altre strutture dati potrebbero essere meno adatte alle esigenze specifiche del progetto. Ad esempio, le liste potrebbero risultare inefficaci per le operazioni di ricerca 
rapida, mentre altre strutture dati complesse, come alberi, grafi o liste di liste, potrebbero aggiungere una complessità non necessaria alla nostra implementazione.\\
\\
Ora verranno descritte le sclete algoritmiche e le loro complessità della parte \texttt{Model} del progetto, in quanto è la sezione che si occupa della gestione dei dati.\\
\\
\textbf{DataStorage}
\begin{enumerate}
    \item Creazione delle Mappe:
    \begin{itemize}
        \item Inizializzazione nel costruttore tramite i metodi \texttt{createCityMap(), createOperatorMap(), createCenterMap(), e createWeatherMap()}.
        \item Ogni metodo legge i dati dal file corrispondente utilizzando \texttt{FileHandler} e costruisce oggetti (es. \texttt{RecordCity, RecordOperator}, ecc.)
        per l'inserimento nella mappa corrispondente.
        \item Complessità: $O(n)$, dove $n$ è il numero di record nel file. 
    \end{itemize}
    \item Operazioni di Ricerca:
    \begin{itemize}
        \item Fornisce metodi (es. \texttt{getCityByID, getOperatorByID, getCenterByID, getWeatherByID}) per recuperare oggetti specifici dalla mappa basandosi sull'ID.
        \item Complessità: $O(1)$, per accesso in una \texttt{HashMap}.
    \end{itemize}
\end{enumerate}

\textbf{DataQuery}
\begin{enumerate}
    \item Metodi di Recupero per Singole Entità:
    \begin{itemize}
        \item Utilizzando il metodo privato\texttt{filterData} per filtrare i dati basandosi su condizioni specifiche.
        \item Complessità: $O(n)$, dove $n$ è il numero di elementi nella collezione.
    \end{itemize}
    \item Metodi di Filtraggio Interni:
    \begin{itemize}
        \item Verificano se un elemento soddisfa una condizione specifica.
        \item Complessità: $O(1)$, ma totale dipende dal numero di condizioni.
    \end{itemize}
    \item Generazione di Epsilon:
    \begin{itemize}
        \item Il metodo \texttt{generateEpsilon} calcola un valore di $\epsilon$ basato sul numero di posizioni deciamli di un dato valore.
        \item Complessità: $O(n)$, dove $n$ è il numero di posizioni decimali.
    \end{itemize}
    \item Caloclo delle Posizioni Decimali:
    \begin{itemize}
        \item Il metodo \texttt{computeDecimalPositions} calcola il numero di posizioni decimali di un valore \texttt{double}.
        \item Complessità: $O(n)$, dove $n$ è la lunghezza della stringa rappresentante il valore.
    \end{itemize}
\end{enumerate}

\textbf{DataHandler}
\begin{enumerate}
    \item Generazione della Chiave Primaria:
    \begin{itemize}
        \item Il metodo \texttt{generatePrimaryKey} cerca iterativamente la chiave $k$ più alta della mappa attuale e genera una nuova chiave $k' = {k+1}$.
        \item Complessità: $O(n)$, dove $n$ è il numero di elementi nella mappa.
    \end{itemize}
    \item Aggiunta di nuovi Record Operatore, Centro e dati Meteorologici:
    \begin{itemize}
        \item Il metodo \texttt{addNewRecord} genera un nuovo record, verifica se ci sono duplicati, aggiunge il record nella relativa mappa e scrive i dati nel file corrispondente.
        \item Complessità: $O(n)$, dove $n$ è il numero di record già presenti.
    \end{itemize}
    Aggiornamento di Record:
    \begin{itemize}
        \item Il metodo \texttt{updateRecord} ricerca il record nel file, lo aggiorna e riscrive il file con il nuovo record.
        \item Complessità: $O(n)$, dove $n$ è il numero totale di righe nel file.
    \end{itemize}
\end{enumerate}

\textbf{LogicCenter}
\begin{enumerate}
    \item Inizializzazione di un nuovo Centro di Monitoraggio:
    \begin{itemize}
        \item Algorimto con controlli su autenticazione, associazione dell'utene a un centro esistente, validità parametri del nuovo centro e verifica validità ID città.
        \item Aggiorna i dati dell'operatore quando viene associato al nuovo centro.
        \item Complessità: $O(n)$, dove $n$ è il numero di città associate al centro.
    \end{itemize}
    \item Aggiunta di dati meteorologici a un centro:
    \begin{itemize}
        \item Controlli sull'autenticazione dell'utente, associazione a un centro, validità della data e dei dati meteorologici.
        \item Aggiorna i dati della mappa e del file corrispondente aggiungendo un nuovo record di dati meteorologici.
        \item Complessità: $O(n)$, dove $n$ è il numero di righe di dati meteorologici fornite.
    \end{itemize}
\end{enumerate}

\textbf{LogicCity}
\begin{enumerate}
    \item Costruttore \texttt{WeatherTableData}:
    \begin{itemize}
        \item Construttore con array di record di dati meteorologici e chiamata a \texttt{processCategory} per ogni categoria presente nei record.
        \item Complessità:$O(n*m)$, dove $n$ è la lunghezza dell'array e $m$ è il numero di categorie.
    \end{itemize}
    \item Metodo \texttt{processCategory}:
    \begin{itemize}
        \item Aggiorna i punteggi, i conteggi dei record e i commenti per la categoria data.
        \item Complessità: $O(1)$ per ogni chiamata.
    \end{itemize}
    \item Metodo \texttt{getCategoryAvgScore}:
    \begin{itemize}
        \item Calcola la media dei punteggi per una categoria data.
        \item Complessità: $O(1)$.
    \end{itemize}
    \item Metodo \texttt{getCategoryRecordCount}:
    \begin{itemize}
        \item Ottiene il conteggio dei record per una categoria data.
        \item Complessità: $O(1)$.
    \end{itemize}
    \item Metodo \texttt{getCategoryComments}:
    \begin{itemize}
        \item Ottine la lista di commenti per una categoria data.
        \item Complessità: $O(1)$.
    \end{itemize}
\end{enumerate}

\textbf{LogicOperator}
\begin{enumerate}
    \item Algoritmo di Login \texttt{performLogin}:
    \begin{itemize}
        \item Verifica se i campi per il login non sono vuoti, esegue il \texttt{logout} se un operatore è già loggato, costruiisce la lista di condizioni per la query da effettuare
        e la esegue per ottenere l'operatore corrispondente alle credenziali fornite.
        \item Complessità: $O(n)$, dove $n$ è il numero di operatori registrati.
    \end{itemize}
    \item Algoritmo di Registrazione \texttt{performRegistration}:
    \begin{itemize}
        \item Controlli di validità su nome, cognome, codice fiscale, e-mail, username e password.
        \item Aggiunge un nuovo record alla mappa corrispondente.
        \item Complessità: $O(1)$, in quanto le operazioni sono indipendenti dal numero di operatori registrati.
    \end{itemize}
    \item Algoritmo di Associazione a un Centro \texttt{associateCenter}:
    \begin{itemize}
        \item Verifica se l'operatore è loggato ed aggiorna l'operatore corrente con l'ID del centro specificato.
        \item Complessità: $O(1)$, in quanto coinvolge solo operazioni di aggiornamento nella mappa.
    \end{itemize}
    \item Algoritmo di Validazione dei Dati:
    \begin{itemize}
        \item Controlli di validità sui dati froniti.
        \item Complessità: $O(n)$, dove $n$ è il numero di campi da controllare.
    \end{itemize}
    \item Algoritmo di Hashing della Password \texttt{hashPassword}:
    \begin{itemize}
        \item E' utilizzato l'algoritmo \texttt{SHA-256} combinato con la concatenazione di username e password.
        \item Complessità: $O(n)$, dove $n$ è la lunghezza dell'input in bit.
    \end{itemize}
\end{enumerate}

\textbf{RecordCity, RecordOperator, RecordCenter, RecordWeather}
\begin{enumerate}
    \item Costruttore del Record:
    \begin{itemize}
        \item Assegna i valori forniti ai campi del record.
        \item Complessità: $O(1)$.
    \end{itemize}
    \item Metodo \texttt{toString}: 
    \begin{itemize}
        \item Restituisce una rappresentazione testuale del record.
        \item Complessità: $O(n)$, dove $n$ è la lunghezza dell'array dei campi.
    \end{itemize}
\end{enumerate}

\textbf{CurrentOperator}
\begin{enumerate}
    \item Pattern Singleton:
    \begin{itemize}
        \item Implementato con una variabile statica privata \texttt{instance} e costruttore privato.
        \item \texttt{getInstance()} restituisce l'instanza unica di \texttt{CurrentOperator}.
        \item Complessità: $O(1)$, coinvolge solo operazioni di controlloe e allocazione di memoria.
    \end{itemize}
    \item Pattern Observer:
    \begin{itemize}
        \item Metodi \texttt{addCurrentUserChangeListener} e \texttt{removeCurrentUserChangeListener} gestiscono l'aggiunte e la rimozione di listener per i cambiamenti di utente.
        $O(1)$ per entrambi.
        \item Metodo \texttt{notifyCurrentUserChange} notifica tutti i listener registrati di un cambio di utente.
        $O(n)$, dove $n$ è il numero di listener registrati.
        \item Metodo \texttt{setCurrentOperator} imposta l'operatore corrente e notidica i listener solo se l'operatore è diverso da quello corrente.
        $O(n)$, per la chiamata a \texttt{notifyCurrentUserChange}.
    \end{itemize}
\end{enumerate}