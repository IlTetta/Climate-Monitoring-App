\NoBgThispage
\section{Introduzione}

Climate Monitoring è un software client-server sviluppato in Java per il Laboratorio A/B del corso in Informatica dell’Università degli Studi dell’Insubria.
Il codice sorgente è stato scritto in Java 17 e il software è stato sviluppato con l’ausilio di Apache Maven per la gestione delle dipendenze e la compilazione del progetto.
Il software è stato sviluppato per la gestione di centri di monitoraggio, che inviano dati in tempo reale al server, il quale si occupa di memorizzarli e di fornirli ai client che ne fanno richiesta.
Per il lato server è stato utilizzato PostgreSQL 42.7.3.jar che permette la connessione ai database PostgreSQL, mentre per il lato client è stato utilizzato JavaSwing per la creazione dell’interfaccia grafica.